
This thesis work mainly focused on drone application with a single drone deployment. 
This limitation clearly does not fit modern HDI research's needs. 

For this reason, a possible future development around the EasyFly programming environment is the possibility of deploying multiple drones without introducing useless complexity in the development phase.
In the actual implementation of the standard cflib there is a possibility of running scripts for swarm of drones.

EasyFly should extend this basic feature by integrating it within its Coordination Manager, allowing for inter-drone communication.

Another key point of future development should be around the simulation environment. 
For the purpose of this work, we used the simulation environment mainly to evaluate performance.
In our opinion, allowing the user to leverage the potentiality of such a system should be critical.

Testing with drones is costly; hence, having the possibility to rely on a solid simulation environment is essential.
A future version of our simulation environment should include the possibility of utilizing all the available sensors inside the simulation. 
Moreover, multiple drone deployments should be allowed inside the simulation to support swarm programming.