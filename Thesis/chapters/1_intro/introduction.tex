In recent years, the rapid advancement of unmanned aerial vehicles, commonly known as drones,
has revolutionized various industries and opened up new possibilities for applications ranging
from aerial photography and package delivery to search and rescue operations. As drones become
increasingly integrated into our daily lives, it is crucial to explore and understand the dynamics
of their interaction with humans.

In modern drone applications, the human figure is marginal with respect to the drone.
The former usually plays the supervisor role, where the main task is to control the activities and check that every operation is carried out correctly. 
At the same time, the latter performs almost all requested tasks automatically.
This significant discrepancy undoubtedly leads to a decrease in interactions between the two.

This thesis will describe EasyFly, an accessible and high-level programming environment for drone applications.
The purpose is to provide a programming environment to allow users with different levels of expertise to experiment with drones. 
Differently from modern applications, our environment allows humans and drones to work closely together,
making EasyFly a perfect tool for conducting research in the field of human-drone interactions.


\section{The Problem: Programming Human-Drone Interactions}\label{sec:the_problem}
Human-Drone Interaction (HDI) is a branch of the more general field of Human-Robot Interaction (HRI), and it can be defined as 
`the study field focused on understanding, designing, and evaluating drone systems for use by or with human user~\cite{tezza2019hdi}`.
While the field of HRI offers valuable insights, the distinctive ability of drones to move freely in three-dimensional
space, along with their unique shapes, sets HDI as a distinct and independent area of research.

The rapid technological progress in this field has made drones increasingly efficient and autonomous in performing various
tasks. While on the one hand, these advancements enable the integration of drones into everyday life, streamlining processes
and reducing the time required for specific activities, on the other hand, modern drone applications do not represent the
ideal prototype for conducting research in the field of HDI.

The first limitation of modern drone applications in this discipline's study is that these applications are designed
to operate in large environments with minimal human presence. An example can be represented by autonomous delivery
drones~\cite{singireddy2018primeAir} or 3D mapping applications~\cite{nex3Dmapping} where the interactions with the human are purposely
reduced to the bare minimum.

In these types of applications, interactions between humans and drones are often limited to simple tasks, such as package
delivery or crop monitoring. These interactions are often repetitive and lack the diversity and complexity required in
research on HDI. Since tasks and interactions are repetitive, users are usually trained to interact with the drone
in a specific way. This training can reduce the variability in HDI, making it less suitable for research purposes.

Programming HDI is usually the field's most complex and expensive task. 
Usually, researchers in this field are unfamiliar with programming at low-level embedded systems like drones.
For this reason, a specialized team of researchers is typically required to program a custom drone application that addresses the complexity of the interactions requested. 

Moreover, the implementation phase is usually the bottleneck of the entire process; every small change to the interaction model can result in days or weeks for implementing the desired behavior.
In fact, one of the most significant challenges during the programming of HDI is the testing phase.
Modern drones are usually fragile and expensive, while tests are likely to fail. 
This phase usually introduces a high consumption of resources, both in terms of costs and time needed for repairing the entire setup before another attempt.


\section{The Solution: EasyFly}\label{sec:the_solution}
To overcome the issues related to the programming of HDI, we introduce EasyFly, a programming environment for drone applications that addresses all the research needs in the field of HDI.

To better understand the contribution of EasyFly to this research field, let us take a step back and describe the needs of researchers.

The ideal prototype of a programming environment for researchers studying HDI should address and solve all the problems related to developing drone applications used for research. 
In particular, this prototype should ultimately reduce the time and costs associated with the development phase and increase the research's effectiveness.

The first characteristic of the ideal prototype is to reduce the level of expertise needed to develop the desired drone application. 
This feature allows researchers to easily implement all the required functionality without requiring a specialized drone programmer team. 
Moreover, this feature would open the doors to a brand-new type of research where the users interact with the drone programmed by themselves.
EasyFly provides this feature by offering a set of simple operations, which indeed allows the creation of very complex behaviors.
In addition to this, EasyFly allows programming in a descriptive fashion; in this way, programs would be self-explaining and easily interpreted by anyone.

As in any other field, research on HDI should be dynamic. 
In other words, to gather all the possible insights from an interaction, the drone application must rapidly change and adapt to the situation.
If the application development cycle is too long, there is the risk of losing many possible opportunities to experiment with possible alternative solutions. 
The ideal prototype should provide the maximum flexibility in adapting to many possible situations.
For this reason, EasyFly has adopted a modular approach for both the hardware and the software components.
At any moment, a module (either software or hardware) can be attached or detached to compose the best configuration needed at that specific moment.

Last but not least, facilitating the interaction between the human and the drone should be the primary goal.
The ideal prototype should be the first promoter of the interaction. 
It should offer the best possible condition to allow the two entities to establish an interaction safely and free from any potential bias determined by the programming environment.
In other words, it should use a typology of drones that allow close contact.

To implement EasyFly, we have targeted a specific typology of unmanned aerial vehicles: nano-drones.
As the name suggests, nano-drones are simply drones with very small size and weight. Their small size makes them the best choice to facilitate human-drone interaction.

In the first place, nano-drones are less intimidating and intrusive than larger drones, making it easier for
researchers to observe how individuals react and interact with them. They also allow minimal disturbance in the
observation environment, making them perfect for avoiding any possible noise in the experiment.
Given that the drone and the human are supposed to work closely together, any possible malfunction can cause an unexpected drone crash, 
especially while experimenting with new solutions. It is easy to deduce that the smaller the drone is, the safer the interaction.
The last observation is that nano-drones are usually less expensive than bigger ones, allowing researchers
to experiment with interactions with multiple drones without affecting their budget.


\section{The Benchmark: Drone Arena Challenge}\label{sec:the_benchmark}
In the HDI domain, the research's core part, especially from the computer science perspective,
is the experimental phase. During this phase, researchers put their ideas and prototypes to test
and assess the practicality of innovative interaction models.

For a programming environment like EasyFly, testing and evaluating in a real research scenario in HDI is essential. 
The testing in real scenarios can help detect possible weaknesses in the programming environment, allowing for fine-tuning the model.

To best evaluate our EasyFly programming environment, we had the possibility to participate in the Digital Futures Drone Arena project~\cite{dronearena}.
This project allowed us to perform an in-depth analysis of the impact of using EasyFly while developing human-drone interactions. 

Drone Arena is an interdisciplinary research project that aims to create a technological and conceptual platform for interdisciplinary investigations of drones at the intersection of mobile robotics,
 autonomous systems, machine learning, and Human-Computer Interaction.
The project has three inter-related objectives:
\begin{enumerate}
    \item   The constructions of a novel aerial drone testbed that is geared towards application-level
            functionality rather than low-level control mechanisms.
    \item   The organization of two challenges where multiple teams are involved and tasked to realize
            functionality that pushes the state of the art.
    \item   Conducting empirical investigation of Human Drone Interactions in the Drone Arena.
            This includes observation, interviews, and micro-sociological video analysis to inform future
            competitions in the drone arena and to develop insights from the movement-based explorations of drone piloting.
\end{enumerate}

In these settings, our EasyFly programming environment is focused on the first two objectives.
In particular, our programming environment was one of the core parts of the novel aerial drone testbed used for the entire duration of the project. 
For the second objective of the project, we had the possibility to actively participate in the first of the two challenges organized for the Drone Arena project. 
During this challenge, we conducted a complete and in-depth evaluation of the impact of using EasyFly; 
in particular, we compared our programming environment with a simpler and lower-level one.


\section{State of the Art}\label{sec:intro_soa}
The field of HDI is an active and evolving area of research with a focus on improving the ways in which humans
and drones interact. It is a multidisciplinary field with two main research areas: technological and sociological.
Each area focuses on distinct aspects of the interaction between humans and drones~\cite{hri2009davidMaya}.

In the technological area, at the intersection of computer science, mobile robotics, autonomous systems, and machine learning,
the key focus is developing and improving the hardware and software components of drones and their interfaces~\cite{kolling2012towards, giusti2012distributed}.
The main goal of this area is to enhance the capabilities and functionality of drones to make them more user-friendly and efficient~\cite{cauchard2015droneAndMe}.

In the sociological area, which includes disciplines like social engineering, art, ethics, and political science,
the core objective is to understand how the presence and use of drones impact society, individuals, and communities~\cite{eriksson2020ethicsInMovement, anderson2012accidentally}.

Especially in research focused on the sociological area, where researchers usually have less familiarity with programming tools, 
the prototyping phase is the most complex and time-consuming. 
EasyFly tries to overcome all the issues related to this phase by creating a simple and flexible programming environment for drone applications.
Moreover, it tries to offer a new perspective in the investigation of human-drone interactions where the user plays the role of the programmer.

\section{Overview}\label{sec:intro_overview}
This section aims to provide a guideline for our thesis work.

In Chapter~\ref{ch:soa}, we will start by giving a global picture of the current state of the art.
To better understand the background of our work, we divided the literature analysis into two main sections. 
The first part is mainly related to the human-drone interaction research, while the second part focuses on the programming environments for drone applications.

In Chapter~\ref{ch:tools}, we give an overview of the set of tools used as a baseline for this work. 
In particular, we will see the fundamentals of the CrazyFlie platform produced by Bitcraze~\cite{bitcraze}.

The Chapter~\ref{ch:ecf} sings the beginning of our contribution. 
In this chapter, we will present the main component of our programming environment: the Extended Crazyflie (ECF).
The ECF acts as a parent container for all the other parts of EasyFly.

We then move on to describe the various parts of the programming environment more in detail. 
In particular, in Chapter~\ref{ch:communication}, we will present the Communication Framework that is the main component to handle the communication flow to and from the Crazyflie.
This Framework is split into two specific implementations, each of which handles a specific direction of the flow of communication. 
The Logging Managers handle the communication from Crazyflie, such as telemetry data.
The Parameters Managers, on the other hand, handle the communication directed to the CrazyFlie, such as configuration parameters.

In Chapter~\ref{ch:coordination}, we will describe the other frameworks of the EasyFly programming environment: the Coordination Framework.
This Framework is an event-based system that allows the developer to solve their application by specifying how to react upon receiving events.
This framework's role is central and allows developers to write simpler solutions with less effort.

In Chapter~\ref{ch:modules}, we present the last part of EasyFly, which is a set of Modules (ECF Modules) that enrich the programming environment capabilities by adding specific functionality that developers can use to implement more clean and faster solutions.
These Modules give the EasyFly programming environment a modularized design suitable for highly dynamic applications. 

We conclude our work in Chapter~\ref{ch:conclusions} with a detailed evaluation of our programming environment.