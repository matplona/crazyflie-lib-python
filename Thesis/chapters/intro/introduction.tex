In recent years, the rapid advancement of unmanned aerial vehicles, commonly known as drones,
has revolutionized various industries and opened up new possibilities for applications ranging
from aerial photography and package delivery to search and rescue operations. As drones become
increasingly integrated into our daily lives, it is crucial to explore and understand the dynamics
of their interaction with humans.

In modern drone applications, the human figure is marginal with respect to the drone.
The former usually plays the supervisor role, while the latter performs almost all requested tasks automatically.
This significant discrepancy undoubtedly leads to a decrease in interactions between the two.

This thesis will describe EasyFly, an accessible and high-level programming environment for drone applications.
The purpose is to provide a programming environment to allow users with different levels of expertise to experiment with drones. 
Differently from modern applications, our environment allows humans and drones to work closely together,
making EasyFly a perfect tool for conducting research on the field of human-drone interactions.


\section{The Problem: Programming Human-Drone Interactions}\label{sec:the_problem}
Human-Drone Interaction (HDI) is a branch of the more general field of Human-Robot Interaction (HRI), and it can be defined as 
`the study field focused on understanding, designing, and evaluating drone systems for use by or with human user`~\cite{tezza2019hdi}.
While the field of HRI offers valuable insights, the distinctive ability of drones to move freely in three-dimensional
space, along with their unique shapes, sets HDI as a distinct and independent area of research.

The rapid technological progress in this field has made drones increasingly efficient and autonomous in performing various
tasks. While on the one hand, these advancements enable the integration of drones into everyday life, streamlining processes
and reducing the time required for specific activities, on the other hand, modern drone applications do not represent the
ideal prototype for conducting research in the field of HDI.

The first limitation of modern drone applications in this discipline's study is that these applications are designed
to operate in large environments with minimal human presence. An example can be represented by autonomous delivery
drones~\cite{singireddy2018primeAir} or 3D mapping applications~\cite{nex3Dmapping} where the interactions with the human are purposely
reduced to the bare minimum.

In these types of applications, interactions between humans and drones are often limited to simple tasks, such as package
delivery or crop monitoring. These interactions are often repetitive and lack the diversity and complexity required in
research on HDI. Since tasks and interactions are repetitive, users are usually trained to interact with the drone
in a specific way. This training can reduce the variability in HDI, making it less suitable for research purposes.

Programming HDI is usually the field's most complex and expensive research phase. It usually requires
a specialized team of researchers who can program a custom drone application that addresses the complexity of interactions
required. Moreover, the implementation phase is usually the bottleneck of the entire process; every small change to
the interaction model can result in days or weeks for implementing the desired behavior.

One of the most significant challenges during the programming of HDI is the testing phase. Modern drones are
usually fragile and expensive, while tests are likely to fail. This phase usually introduces a high consumption of resources, 
both in terms of costs and time needed for repairing the entire setup before another attempt.


\section{The Solution: EasyFly}\label{sec:the_solution}
To overcome all the issues related to the programming of HDI, we introduce EasyFly, a programming environment for
drone applications that addresses all the needs of research in the field of HDI.

To better understand the contribution of EasyFly to this research field, let us make a step back in describing
what are the needs of researchers.

The ideal prototype of a programming environment for researchers studying HDI, should address and solve all the problems
related to the development of drone applications used for research. In particular, this prototype should ultimately be able
to reduce the time and costs associated with the development phase along with increasing the effectiveness in the research.

The first characteristic of the ideal prototype is to reduce the level of expertise needed to develop the desired drone
application. This feature allows researchers to easily implement all the functionality required, without the needs for
a specialized team of drone programmers. Moreover, this feature would open the doors to a brand-new type of research where
the users are questioned to interact with the drone programmed by themselves.
EasyFly provides this feature by offering a set of simple operations, which indeed allows building very complex behaviors.
In addition to this, EasyFly allows programming in a descriptive fashion, in this way programs would be self explaining
and easily interpreted by anyone.

As in any other filed, also research on HDI should be dynamic. In other words, to gather all the possible
insights from an interaction, the drone application needs to rapidly change and best adapt to the situation.
If the application developments cycle is too long, there is the risk to lose many possible opportunities to
experiments possible alternative solutions. The ideal prototype should provide the maximum flexibility in adapting
to many possible situations.
For this reason EasyFly has adopted a modular approach for both the hardware and the software components.
At any moment, a module (either software or hardware) can be attached or detached to compose the best configuration
needed in that specific moment.

Last but not least, the facilitation of the interaction the between human and the drone should be the primary goal.
The ideal prototype should be the first promoter of the interaction. It should offer the best possible condition to
allow the two entities to establish an interaction safely and free from any potential bias determined by the programming environment.\\
For the implementation of EasyFly we have targeted a specific typology of unmanned aerial vehicles: nano-drones.
As the name suggest, nano-drones are simply drones with very small size and weight. Given their small size,
they are the best choice to facilitate human-drone interaction.\\
In first place nano-drones are less intimidating and intrusive compared to larger drones, this can make it easier for
researchers to observe how individuals react and interact with them. They also allow a minimal disturbance in the
environment of the observation, making them perfect for avoiding any possible noise in the experiment.
Given the fact that drone and human are supposed to work closely together and that any possible malfunctioning can
cause an unexpected crash of the drone, especially while experimenting new solutions, it is easy to deduce that smaller
the drone is, safer is the interaction.\\
Last observation is that nano-drones are usually less expensive with respect to bigger ones. This allows researchers
to experiment interactions with multiple drones without affecting their budget.


\section{The Benchmark: Drone Arena Challenge}\label{sec:the_benchmark}
In the domain of HDI, the core part of the research, especially from the computer science perspective,
is the experimental phase. It is during this phase that researchers put their ideas and prototypes to test
and asses the practicality of innovative interaction models.

For a programming environment like EasyFly, it is essential to be tested and evaluated in a real scenario of
research in HDI. The usage in real scenarios can help in detecting possible weaknesses of the programming environment,
allowing for a fine tune of the model.

To best evaluate our EasyFly programming environment, we had the possibility to participate in the
Digital Futures Drone Arena project. This project allowed us to perform an in depth analysis on the impact of
using EasyFly while developing human-drone interactions.

Drone Arena is an interdisciplinary research project which aims to create a technological and conceptual platform for
interdisciplinary investigations of drones at the intersection of mobile robotics, autonomous systems, machine learning,
and Human Computer Interaction.
The project has three inter-related objectives:
\begin{enumerate}
    \item   The constructions of a novel aerial drone testbed that is geared towards application-level
            functionality rather than low-level control mechanisms.
    \item   The organization of two challenges where multiple teams are involved and tasked to realize
            functionality that push the state of the art.
    \item   The conduction of empirical investigation of Human Drone Interactions in the Drone Arena.
            This includes observation, interviews, and microsociological video analysis, to inform future
            competitions in the drone arena and to develop insights from the movement-based explorations of drone piloting.
\end{enumerate}

In these settings, our EasyFly programming environment is focused on the first two objectives of the project.
In particular, our programming environment was one of the core part of the novel aerial drone testbed
used for the entire duration of the project. 
For the second objective of the project we had the possibility to e had the opportunity to actively participate
in the first of the two challenges organized for the Drone Arena project. During this challenge we conducted
a complete and in depth evaluation on the impact of using EasyFly, in particular we compared our programming
environment with another that is simpler and lower-level.



\section{State of the Art}\label{sec:intro_soa}
The field of HDI is an active and evolving area of research with a focus on improving the ways in which humans
and drones interact. It is a multidisciplinary filed with two main area of research, the technological and the sociological.
Each area focuses on distinct aspects of the interaction between humans and drones.

In the technological area, at the intersection of computer science, mobile robotics, autonomous systems and machine learning,
the key focus is on the development and improvement of the hardware and software components of drones and their interfaces.
The main goal of this area is to enhance the capabilities and functionalities of drones to make them more user-friendly and efficient.

In the sociological area, which includes disciplines like social engineering, art, ethics and political science,
the core objective is to understand how the presence and use of drones impact society, individuals, and communities.

The most commonly used drone in researches on HDI filed, is the Parrot ARDrone \cite{tezza2019hdi}. 
Parrot drones provide an easy-to-use software API allowing for quick prototyping, which is likely the reason they are the
researcher's first choice. Although the Parrot ARDrone is widely used for research,
this model was discontinued by the manufacturer.

Especially in researches focused on the sociological area, where researcher has usually less familiarity with programming tools, 
the progr

\begin{itemize}
    \item currently the state of the art is divided in programming environments for drone, and \hdi{} researches.
    \item \hdi{} researcher usually uses parrot or dij as tool for their project
    \item there is always a team of drone experts that builds a custom applications setup for every \hdi{} research.
    \item problem can be:
          \begin{itemize}
              \item difficulties in adapting to many different situation for a single application
              \item static testing environment
          \end{itemize}
    \item \easyfly{} tires to output an easy environment that allows for two kind of research:
          \begin{itemize}
              \item researchers can build and change the application autonomously
              \item researcher are allowed to investigate in how users program, and uses drones
          \end{itemize}
\end{itemize}



\section{Overview}\label{sec:intro_overview}