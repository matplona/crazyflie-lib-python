As described in the previous chapter, ECF Modules are the most essential part of the EasyFly programming environment. 
In this chapter, we will analyze each ECF Module available in the EasyFly extension in detail, describing the purpose, main functionalities, and implementation details for each. 
We will first analyze the Modules with an internal state and then move to the Modules with a sensed data state. 

\section{State Estimate Module}\label{sec:module_state_estimate}

Regardless of the purpose, a crucial aspect of operating a drone is having a deep understanding of the drone's position, velocity, and acceleration. 
These parameters are vital for ensuring the drone's stability and safe navigation. 
Accurate knowledge of the drone's position and speed enables its operator to control its movements and avoid obstacles or hazards in its path. 
Furthermore, understanding acceleration is crucial in achieving precision and accurate control of the drone's movements.

As the name suggests, the State Estimate Module is an ECF Module that manages the estimated variables of the Crazyflie 2.1.
The full state of the State Estimate ECF Module is:
\begin{lstlisting}[language=Python]
State = {
    "x" : "Position on the x axis from the origin",
    "y" : "Position on the y axis from the origin",
    "z" : "Position on the z axis from the origin",
    "vx" : "Velocity on the x axis",
    "vy" : "Velocity on the y axis",
    "vz" : "Velocity on the z axis",
    "ax" : "Acceleration on the x axis",
    "ay" : "Acceleration on the y axis",
    "az" : "Acceleration on the z axis",
    "roll" : "Roll in rad",
    "pitch" : "Pitch in rad",
    "yaw" : "Yaw in rad",
    "rateRoll" : "Roll rate in rad/s",
    "ratePitch" : "Pitch rate in rad/s",
    "rateYaw" : "Yaw rate in rad/s",
}
\end{lstlisting}

Because the control loop onboard the Crazyflie 2.1 runs at 500Hz, a new estimation is computed every two milliseconds; these variables can rapidly change over time. 
We need to update the state as frequently as possible to ensure that the information provided is accurate and precise. 
For this reason, we selected the lowest possible sampling period for the Communication Framework, which is 10 milliseconds.


In this ECF Module, the utility functions allow the user to record the state variable and, at the end of the flight, plot the recorded data.
As with any other telemetry data, it is usually vital for a user to have the possibility to analyze that information after the flight, especially in the development phase of the application.
For this reason, the utility functions that we implemented will allow a user to analyze and compare data from multiple application runs.


\section{Battery Module Module}\label{sec:module_battery}

When dealing with drones, battery management is an important factor that must always be considered. 
Usually, due to its weight, the battery has minimal capacity, especially on drones with small dimensions.

An application can be perfectly developed, but it can easily fail if it does not consider the limitation of resources like the battery.
To help the developer with this task, we created another ECF Module that manages battery-related information and keeps them updated.

Given this, the full state of the Battery ECF Module is:
\begin{lstlisting}[language=Python]
State = {
	"pm_state" : "Battery power management state",
	"voltage" : "Battery voltage in V",
	"battery_level" : "Estimated battery level in percentage",
}
\end{lstlisting}

The first variable of the state, \textit{pm\_state}, represents the state in which the power management of the drone is in.
The state can be one of the following:
\begin{itemize}
    \item Battery -- The drone is on and using its battery.
    \item Charging -- The drone is plugged into the power supply.
    \item Charged -- The drone has completed the recharge, and its battery is fully charged.
    \item Low Power -- The drone needs to be recharged.
    \item Shutdown -- The drone is off.
\end{itemize}

With the other two properties of the state, \textit{voltage} and \textit{battery\_level}, the developer can determine with high precision the recharging phase.

Since battery management strictly depends on the application to be developed, it is hard to find an implementation that satisfies all the possible usage.
For this reason, we decided to leave the implementation of the decision processes for determining the outside of the Battery Module. 
The module is responsible only for keeping the information in the state consistent and updated.

\section{Multiranger Module}\label{sec:module_multiranger}



\section{Z ranger Module}\label{sec:module_zranger}



\section{Lighthouse Module}\label{sec:module_lighthouse}



\section{AI deck Module}\label{sec:module_ai_deck}